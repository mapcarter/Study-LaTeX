\newpage

{\let\clearpage\relax \chapter{版面和格式}}

\section{页眉页脚}
一般来说,设置页眉页脚需要调用使用比较广泛的\qd{fancyhdr}宏包。我习惯使用如下代码先清空默认定义,然后自己重新定义。页眉页脚线的粗细也可以重新定义。

\begin{latex}{}
\usepackage{fancyhdr}
\pagestyle{fancy}
\fancyhf{}					%清空当前设置
	%单页文档
	\lhead{}				%l,r,c,左中右
	\cfoot{}
	%双页文档
	\fancyhead[RO,LE]{}		%E,O,左、右页
	\fancyfoot[LE,RO]{\thepage}
\renewcommand{\headrulewidth}{0.4 pt}
\renewcommand{\footrulewidth}{0.4 pt}
\end{latex}


我们可以将章节标题和序号插入到页眉或者页脚中去,其格式与正文中章节标题的定义一样。如果需要更改,要重新定义。例如,可以使用如下代码重新定义页眉内的章标题样式,用在在本书中,这将会使页眉的“第X章 版式”更改为“X 版式”。 

具体更改页眉页脚区域章节显示样式的代码如下。

\begin{latex}{}
\renewcommand{\chaptermark}[1]{\markleft{\thesection.\#1}}
%两种一样,\markleft影响\leftmark,而\makeboth影响两着,需要选一
\renewcommand{\chaptermark}[1]{\markboth{\thechapter.\ #1}{节样式空置表示修改章样式}}
\renewcommand{\chaptermark}[1]{\markboth{章样式}{节样式}}
\end{latex}


在book文件类别下,\qd{\leftmark}自动存录各章之章名,\qd{\rightmark}记录节标题。所以,想要在页眉上显示章节标题是很容易实现的。


\begin{latex}{}
\lhead{\leftmark}			%左页眉显示章
\rhead{\rightmark}			%右页眉显示节
\end{latex}


\section{代码展示}
我们需要使用一个宏包——\qd{lstlisting}来达到这个目的。记住,不要再\qd{lstlisting}环境的第一行代码前面使用Tab或者空格,因为会无条件还原,导致代码不能贴合框线,丑陋。

MATLAB代码高亮测试


\begin{lstlisting}[language=Matlab]
t=0:pi/10:2*pi;
[X,Y,Z]=cylinder(4*cos(t));
subplot(1,2,1);mesh(X);title('X');
subplot(1,2,2);mesh(Y);title('Y');
\end{lstlisting}


\section{字体设置}

分全局和局部字体设置。
\subsection{全局字体设置}
中文的文档都要调用\qd{ctex}宏包,该宏包提供一个简单的参数可以设置全部正文的字体。

\begin{latex}{}
\setmainfont{Times New Roman}	%仅对西文起作用
\setCJKmainfont{SimSun}			%仅对中文起作用
\end{latex}

\subsection{局部字体设置}

\begin{latex}{}
\newfontfamily\daima{Consolas}	%使用\daima直接调用
\end{latex}

\subsection{在数学环境中使用中文}
默认情况下,数学环境中是不允许输入汉字的。当我们需要输入汉字作为变量的标识时,可以使用\latexline{\text{要输入的汉字字符}}来完成这项工作。

\begin{codeshow}
$f_{\text{高温}}$
\end{codeshow}

\subsection{汉字“斜体”}
汉字没有加斜体。平常我们看到的加斜汉字,通常是几何变换得到的结果,非常的粗糙,并不严格满足排版要求;而真正的字形是需要精细的设计的。同时,汉字字体里面也很少有加粗体的设计。但是,有时候却又有所谓的“斜体”需求。\LaTeX 也是可以实现这种伪斜体的。

\begin{center}
	{\CJKfontspec[FakeSlant=0.4]{SimSun}\zihao{1}汉字伪斜体}
\end{center}

\begin{latex}{}
{\CJKfontspec[FakeSlant=0.4]{SimSun}\zihao{1}汉字伪斜体}
\end{latex}

西文一般设有加斜,但是这与“斜体”并不是同一回事。加斜是指某种字族的Italy 字系;而斜体,是指Slant 字族。所以\latexline{textit{}}是加斜字体,而\latexline{textsl{}}或者\latexline{slshape}才是真正的斜体。

\begin{center}
	{\zihao{1} \textit{Italy Slant} ~ \textsl{Italy Slant}}
\end{center}

\section{颜色}
一般来说,我们调用下\qd{xcolor}这个宏包。如果对内置的颜色了解,或者现有\qd{RGB}颜色值,一般使用如下代码直接调用颜色。

\begin{center}
	\color[RGB]{204, 128, 92}{Color Text中文测试}
\end{center}

\begin{latex}{}
\color[RGB]{204, 128, 92}{Color Text中文测试}
\end{latex}

但是每次调用颜色都写颜色代码似乎不方便,我们可以先定义,基本定义形式如下。
\begin{latex}{}
\usepackage{xcolor}									%颜色宏包
\definecolor{backcolor}{RGB}{242,242,242}			%背景色
\definecolor{comment}{RGB}{0,128,0}					%注释
\definecolor{keyword}{RGB}{0,0,255}					%关键词
\definecolor{name名字随意}{model色值类型}{color-spec色值范围}
\end{latex}

然后,我们就可以直接调用我们定义的颜色名称来设定颜色了。

\begin{center}
	\color{keyword}{\slshape\zihao{3} function, return, if, true, false}
\end{center}

\begin{latex}{}
\color{keyword}{\slshape function, return, if, true, false}
\end{latex}

\section{文本格式}

\subsection{下画线}

\LaTeX 自身提供的下划线之类的命令并不好用,我们使用\qd{ulem}或者\qd{CJKfntef}宏包来代替。对于中文排版,在\qd{ctex}宏包的调用命令中添加\qd{fntef}选项即可调用\qd{CJKfntef}宏包来画线,\qd{CJKfntef}宏包的优点是其画线命令中的中文能自动中断正确换行。本书以\qd{CJKfntef}宏包为例。

\begin{codeshow}
\CJKunderdot{important 非常重要}\\
\CJKunderline{notice 注意}\\
\CJKunderdblline{urgent 必须}\\
\CJKunderwave{prompt 提示}\\
\CJKsout{wrong 错误}\\
\CJKxout{removed 删除}
\end{codeshow}

对于前四种下划线,该宏包还提供了调节下划线与文本之间距离的参数。默认值都是0.2em,需要使用重新定义命令来使参数生效。

\begin{latex}{}
\CJKunderdotbasesep
\CJKunderlinebasesep
\CJKunderdbllinebasesep
\CJKunderwavebasesep
\renewcommand{\CJKunderlinebasesep}{0.5em}
\end{latex}

同时,还提供了6条颜色命令。大致命令形式和调用方法如下所示,其中\qd{color}命令有多种调用模式,详见本书颜色部分。

\begin{latex}{}
\CJKunderlinecolor
\renewcommand{\CJKunderline}{\color{blue}}
\end{latex}

